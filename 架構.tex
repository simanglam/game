\documentclass[12pt, a4paper]{NGPLB}

\usepackage{xeCJK}
\setCJKmainfont{原之味宋體 TW}

\usepackage{tikz}
\usepackage{tcolorbox}
\usepackage{pgf-umlcd}

\usepackage{minted}

\title{遊戲的架構}
\author{Si manglam}

\linespread{1.35}

\begin{document}
\pagenumbering{arabic}

\chapter{概覽}

首先,需介紹這次專題與目標。這次的專題是工學院的跨領域專題,我們的目標是要做出一款遊戲,並且加上我們自製的搖桿,希望可以打破第四道牆,讓搖桿除了操控與震動外還可以有更多的回饋。

整個遊戲檔案被我分成以下的樣子:

\begin{tcolorbox}
\begin{minted}{text}
|------game
    |------core
        |------__init__.py
        |------map
        |------charater
        |-----scene
        |------utilities
        |------updater
            |------__init__.py
\end{minted}
\end{tcolorbox}

每個資料夾各為一系統,例:戰鬥、物理與碰撞,將其分的這麼細緻則是希望可以提高程式重複使用率。每個系統皆為許多子系統組合而成,如 updater 是指遊戲更新處理系統,但一個遊戲更新包含許多不同方面,包含物理、狀態與碰撞。我們將這幾個子系統分開,就可依據不同要求包裝出不同系統,像是戰鬥系統包含碰撞檢測、物理引擎即狀態更新,如果將這幾個子系統分開撰寫,那之後想生出新系統就可以快速地用這邊寫好的東西快速寫出來。

\section{map}

地圖系統負責一件簡單的事,地圖場景地形處理。在我的設想中,地形會被這裡所處理,並渲染至畫面。這邊也會負責定義所有地圖會用到的物件,像土、水與草地等地圖組成元素,這些都會在這一邊定義。

另外,如果未來地圖變得過大,這邊也會負責 Chunk 系統的功能,讓一次刷新的壓力不要那麼大。

\section{charater}

顧名思義,這裡負責定義所有角色。每個角色都會繼承一個父物件,並提供  update method 來被遊戲控制流程呼叫,每個角色都會有自己的一套更新邏輯,並儲存自己的狀態。

至於 render 則是交由其他事物去 render。

\section{scene}

這是負責定義所有場景,所有場景皆有 update 與 render method 可供呼叫。實際上的顯示與遊戲介面上的按鈕與控制邏輯每個場景都不一樣。另外,場景還需提供 change method,只要呼叫這個 method 就可以把當修改前場景。

\section{utilities}

這是負責處理支援功能的,例如渲染位置,我們習慣會以圖片的中心點作為基準點,但 pygame 是以圖片的左上角為基準點,這時我們就可以寫一個 function 去處理這個轉換,而不需用人腦心算。

\section{updater}

updater 是專門處理遊戲中的各種更新的,例如物理引擎、戰鬥引擎,碰撞偵測與狀態更新。這些全部都被放在這個檔案夾理面,並依照需求來裝配出所需的引擎可供使用。

\end{document}